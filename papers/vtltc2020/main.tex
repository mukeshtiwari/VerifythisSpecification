\documentclass{llncs}
\usepackage[utf8]{inputenc}
\usepackage[T1]{fontenc}
\usepackage[scaled=0.8]{beramono}
\usepackage{amsmath}
\usepackage{amssymb}
\usepackage{stmaryrd}
\usepackage{mathtools}
\usepackage{proof}
\usepackage{todonotes}
\usepackage{xspace}

% \usepackage{minted}
\usepackage[colorlinks=true,allcolors=blue]{hyperref}


\title{Verifying the Security of a PGP Keyserver \\
       (Abstract for the VerifyThis challenge~2020)}
\author{Gidon Ernst \inst{1} \and Toby Murray \inst{2} \and Mukesh Tiwari \inst{2}}
\institute{LMU Munich, \email{gidon.ernst@lmu.de} \and
  University of Melbourne, Australia, \email{firstname.lastname@unimelb.edu.au}}

\newcommand{\SecCSL}{\textsc{SecCSL}\xspace}
\newcommand{\SecC}{\textsc{SecC}\xspace}

\begin{document}
\maketitle

\section{Introduction}

We discuss our approach and progress concerning the formal verification of
the HAGRID PGP keyserver, as part of the VerifyThis 2020 Collaborative
Long-term Verification Challenge.

\section{Approach}

We have followed an iterative approach to the formalisation of the case study
and its verification, building on the work of related
teams~\cite{Ernst_Rieger_20}. Specifically, we took Ernst and Rieger's
Scala model of the key server as a starting point. This model represents
the state of the key server as a collection of maps, and has a small
number of top-level functions that manipulate these maps and simulate
actions like sending emails.

\subsection{Abstract Specification}
We began by constructing an abstract specification for the Scala model,
in the spirit of typical Alloy specifications~\cite{jackson2012software}.
Here, the state is modelled as a collection of partial functions, each of
which represents a map in the Scala model. For instance, the \emph{keys}
map stores the uploaded keys, indexed by their fingerprints; the
\emph{uploaded} map remembers which keys have been uploaded and is
indexed by the token that was issued for each; the \emph{prev-tokens} set
remembers previously issued tokens (in order to specify that newly generated
tokens should be fresh).

\medskip
\noindent\begin{tabular}{l}
\textbf{record} \emph{state} = \\
\ \ \emph{keys} :: \emph{fingerprint} $\rightharpoonup$ \emph{key} \\
\ \ \emph{uploaded} :: \emph{token} $\rightharpoonup$ \emph{fingerprint} \\
\ \ \emph{prev-tokens} :: \emph{token} set\\
\ \ \ldots
\end{tabular}
\medskip

Top-level operations of the
Scala model are then specified as relations on these states, describing
how the state after the operation is related to the state before the
operation. These specifications are carefully written to delineate
preconditions and postconditions.

For instance, the precondition for the \emph{upload} operation to
upload a key \emph{k} is a predicate on the pre-state~\emph{s}, and states
that if a key with the same fingerprint of~\emph{k} has already been uploaded, then that key must be identical to~\emph{k}:

\medskip
\noindent\begin{tabular}{l}
  \emph{upload-pre}(\emph{k},\emph{s}) $\equiv$ \emph{k.fingerprint} $\in$ \emph{dom}(\emph{s.keys}) $\implies$ \emph{s.keys}(\emph{k.fingerprint}) = \emph{k} \\
  \end{tabular}
\medskip

The postcondition for the \emph{upload} operation requires that a fresh token
is generated for the key and that the key is added to the \emph{keys} and
\emph{uploaded} maps.

\newcommand{\union}{\cup}
\medskip
\noindent\begin{tabular}{ll}
\emph{upload-post}(\emph{k},\emph{s},\emph{s'}) $\equiv$ $\exists$ \emph{t}. & \emph{t} $\notin$ \emph{s.prev-tokens}  $\land$ \\
& \emph{s'.keys} = \emph{s.keys} $\union$ (\emph{k.fingerprint} $\mapsto$ \emph{k})  $\land$ \\
& \emph{s'.uploaded} = \emph{s.uploaded} $\union$ (\emph{t} $\mapsto$ \emph{k.fingerprint})  $\land$ \\
& \emph{s'.prev-tokens} = \emph{s.prev-tokens} $\union$ \{\emph{t}\}
 \\
  \end{tabular}
\medskip

Having delineated their pre- and post-conditions, operations are specified
straightforwardly. For instance, the specification of the \emph{upload}
operation for uploading a key~\emph{k}, given pre- and post-states
\emph{s} and~\emph{s'} respectively, is:

\medskip
\noindent\begin{tabular}{l}
\emph{upload}(\emph{k},\emph{s},\emph{s'}) $\equiv$ \textbf{if} \emph{upload-pre}(\emph{k},\emph{s}) \textbf{then} \emph{upload-post}(\emph{k},\emph{s},\emph{s'}) \textbf{else} \emph{s' = s}
  \end{tabular}
\medskip

We have experimented with encoding this abstract specification in both
Coq and Isabelle/HOL, and with formalising and proving invariant preservation
over its operations.

\subsection{Verified Model}

While aiding clarity, the purpose of
delineating pre- and post-conditions in the abstract specification was to
serve as a guide for subsequent Hoare logic reasoning about the system.
Specifically, while currently still in progress, the next step of our approach
involves constructing a model of the system
that refines the abstract specification and about which we can reason using
a Hoare logic style program verifier.

In our approach, we chose to use the prototype \SecC verifier, which
automates program reasoning in the Hoare style logic Security Concurrent
Separation Logic (\SecCSL)~\cite{Ernst_Murray_19}. \SecCSL is a variant of
Concurrent Separation Logic that allows reasoning about expressive
information flow security policies, in addition to ordinary Hoare logic
reasoning.

\paragraph{Information Flow Security Policies}

Such policies abound in the key server. For instance, when returning
results pertaining to the lookup of a key~\emph{k}, data about all other keys
should be considered private and not revealed. Additionally, however, whether
a particular identity \emph{id} $\in$ \emph{k.ids}, attached to the
key~\emph{k} should be revealed depends on whether that identity has been
confirmed. Thus only confirmed identities for the key~\emph{k} should be
considered public. The resulting information flow policy for this seemingly
simple operation is therefore highly state-dependent. \SecCSL and \SecC
provide natural support for verifying such stateful policies.


\paragraph{Dynamic Declassification}

Note, however, that the security policy is not fixed: the action of
confirming an identity associated with key~\emph{k} effectively
\emph{declassifies}~\cite{Sabelfeld_Sands_09} the resulting identity,
making it public with respect to a lookup for key~\emph{k}.

\SecC supports this style of reasoning via \textbf{assume} statements.
Whereas traditional Hoare logic verifiers are able to assume only
functional properties, in \SecCSL such assumptions can naturally encompass
security assertions. For instance the statement ``\mbox{\textbf{assume} (\emph{e}::\texttt{low})}'' literally means ``let us assume that the data contained
in the expression~\emph{e} is known to the attacker''~\cite{chudnov2018assuming}. 

\paragraph{Ensuring Correct Declassification}

Assume statements are therefore powerful for reasoning about dynamic
declassification policies. However, we also need to make sure that they
are not mis-used. For example, it is appropriate to place
an \textbf{assume} statement at the
point that an identity~\emph{id} is confirmed for key~\emph{k}, which
declassifies \emph{id} with respect to lookups of key~\emph{k}. However,
the \textbf{assume} statement would not be appropriate if it declassified
\emph{id} with respect to some other key~\emph{k'} or if it did so before
\emph{id} was confirmed. 
Therefore, how can we make sure that
\textbf{assume} statements are used correctly to encode the desired
declassification policy?

To address this issue, our methodology encodes the declassification policy via an
extensional predicate that talks just about the program's inputs and outputs
(i.e.\ not about its internal state), inspired by~\cite{Schoepe_MS_20}.
Then at the site of each \textbf{assume} statement, we immediately
precede the \textbf{assume} statement with an \textbf{assert} statement
to check
that the extensional declassification predicate holds (i.e.\ allows the
declassification encoded in the \textbf{assume} statement to occur).

Doing so allows us to verify expressive declassification policies
naturally via \textbf{assume} statements, free of the risk that we
inadvertently verify the model against the wrong policy.


\bibliographystyle{splncs04}
\bibliography{references}

\end{document}

%%  LocalWords:  HAGRID PGP keyserver VerifyThis Rieger dom t Hoare
%%  LocalWords:  mis
